% Appendix describes how to compute the domain "econ"

\newpage
%\appendix
\begin{center}
	\large Appendix
\end{center}
\setcounter{section}{0}
\renewcommand{\thesection}{Appendix \arabic{section}}   

\section{Computing domain-specific frequencies}
\newcommand{\anaearep}{665}
\newcommand{\anaearepp}{546}
\newcommand{\asumado}{17471}
\newcommand{\asumdo}{13529}
\newcommand{\asumlines}{3722306}
\newcommand{\aadocount}{32}
\newcommand{\adocount}{25}
\newcommand{\adolines}{6817}


Instead of using the output of \texttt{ssc whatshot} to classify strings into likely package names, we have also allowed for the use of domain-specific frequencies. Such frequencies need to come from a large corpus of verified code. In this appendix, we describe how we computed these frequencies for the economics domain (invoked through option \texttt{domain(econ)}).

As part of the AEA Data Editors work, students download replication packages and attempt to run the code in combination with the available data. They do so in a (nearly) clean environment, by invoking the following commands at the start of the command execution:

\begin{stlog}
/* install any packages locally */
capture mkdir "$rootdir/ado"
sysdir set PERSONAL "$rootdir/ado/personal"
sysdir set PLUS     "$rootdir/ado/plus"
sysdir set SITE     "$rootdir/ado/site"
local ssc\_packages "pkg1 pkg2"
{\smallskip}
if !missing("`ssc\_packages'") {\lbr}
        foreach pkg in `ssc\_packages' {\lbr}
                capture which `pkg'
                if _rc == 111 {\lbr}                 
                        dis "Installing `pkg'"
                        ssc install `pkg', replace
                {\rbr}
                which `pkg'
        {\rbr}
{\rbr}
\nullskip
\end{stlog}

These commands should ensure that all necessary packages must be installed into a project specific directory. In rare cases, replication packages may already include \texttt{ado} files, which should then be used. Once installed, the \texttt{ado} \textit{directory} should also be committed to the git repository tracking the reproducibility check.\footnote{Compliance with these instructions is not perfect.} The resulting updated (private) repository thus contains authors' Stata code, and the replicator's installed \texttt{ado} files. 

We thus obtained from our internal records the last known state of \anaearep{} repositories as of July 2021. Of these, \anaearepp{} contained at least one Stata do-file. The average replication package had \adocount{} do-files, and after verification had \aadocount{} ado-files. For each replication package, we ran (an earlier version of) the \texttt{packagesearch} command, and matched against the list of ado-files found. We counted as a \texttt{hit} a package that was confirmed to have the package installed, and counted the hits. We then ranked the list by hits. The resulting list is naturally somewhat different than the \texttt{whatshot} list, and in particular is pruned of spurious package names. Table~\ref{tab:econ} displays the top ten by the \texttt{domain(econ)} ranking, Table~\ref{tab:whatshot} displays the top ten packages per \texttt{whatshot} and their equivalent \texttt{domain(econ)} ranking.

\begin{table}[ht!]
		\centering
		\caption{Top ten \texttt{econ} packages.
		\label{tab:econ}}
	\begin{threeparttable}
		
\begin{tabular}{lrr}\toprule
Packagename&\texttt{econ}&\texttt{whatshot}\\\midrule
estout&1&2\\
reghdfe&2&5\\
outreg2&3&1\\
coefplot&4&13\\
ftools&5&6\\
ivreg2&6&8\\
ranktest&7&16\\
binscatter&8&24\\
spmap&9&25\\
unique&10&20\\
\bottomrule\end{tabular}


		\begin{tablenotes}
			\item \footnotesize Generated on 2022-01-30.
		\end{tablenotes}
	\end{threeparttable}
\end{table}

\begin{table}[ht!]
		\centering
		\caption{Top ten \texttt{whatshot} packages.}
		\label{tab:whatshot}
	\begin{threeparttable}
		
	\begin{tabular}{lrr}\toprule
Packagename&\texttt{econ}&\texttt{whatshot}\\\midrule
outreg2&3&1\\
estout&1&2\\
asdoc&&3\\
winsor2&51&4\\
reghdfe&2&5\\
ftools&5&6\\
logout&119&7\\
ivreg2&6&8\\
ivreg29&&9\\
ivreg210&&10\\
\bottomrule\end{tabular}
	
		
		\begin{tablenotes}
			\item \footnotesize Generated on 2022-01-30.
		\end{tablenotes}
	\end{threeparttable}
\end{table}

