% Appendix describes how to compute the domain "econ"

\newpage
\appendix

\section{Computing domain-specific frequencies}
\input{appendixa-strings}

Instead of using the output of \texttt{ssc whatshot} to classify strings into likely package names, we have also allowed for the use of domain-specific frequencies. Such frequencies need to come from a large corpus of verified code. In this appendix, we describe how we computed these frequencies for the economics domain (invoked through option \texttt{domain(econ)}).

As part of the AEA Data Editors work, students download replication packages and attempt to run the code in combination with the available data. They do so in a (nearly) clean environment, by invoking the following commands at the start of the command execution:

\begin{stlog}
/* install any packages locally */
capture mkdir "$rootdir/ado"
sysdir set PERSONAL "$rootdir/ado/personal"
sysdir set PLUS     "$rootdir/ado/plus"
sysdir set SITE     "$rootdir/ado/site"
local ssc\_packages "pkg1 pkg2"
{\smallskip}
if !missing("`ssc\_packages'") {\lbr}
        foreach pkg in `ssc\_packages' {\lbr}
                capture which `pkg'
                if _rc == 111 {\lbr}                 
                        dis "Installing `pkg'"
                        ssc install `pkg', replace
                {\rbr}
                which `pkg'
        {\rbr}
{\rbr}
\nullskip
\end{stlog}

These commands should ensure that all necessary packages must be installed into a project specific directory. In rare cases, replication packages may already include \texttt{ado} files, which should then be used. Once installed, the \texttt{ado} \textit{directory} should also be committed to the git repository tracking the reproducibility check.\footnote{Compliance with these instructions is not perfect.} The resulting updated (private) repository thus contains authors' Stata code, and the replicator's installed \texttt{ado} files. 

We thus obtained from our internal records the last known state of \anaearep{} repositories as of July 2021. Of these, \anaearepp{} contained at least one Stata do-file. The average replication package had \adocount{} do-files, and after verification had \aadocount{} ado-files. For each replication package, we ran (an earlier version of) the \texttt{packagesearch} command, and matched against the list of ado-files found. We counted as a \texttt{hit} a package that was confirmed to have the package installed, and counted the hits. We then ranked the list by hits. The resulting list is naturally somewhat different than the \texttt{whatshot} list, and in particular is pruned of spurious package names. 

